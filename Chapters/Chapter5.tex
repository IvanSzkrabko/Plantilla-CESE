% Chapter Template

\chapter{Conclusiones} % Main chapter title

\label{Chapter5} % Change X to a consecutive number; for referencing this chapter elsewhere, use \ref{ChapterX}


%----------------------------------------------------------------------------------------

%----------------------------------------------------------------------------------------
%	SECTION 1
%----------------------------------------------------------------------------------------

\section{Resultados obtenidos}

Fue muy importante la planificación inicial para enmarcar y organizar el trabajo. Las estimaciones fueron correctas, aunque se subestimó la etapa de integración de todos los componentes. El trabajo logró integrar realidad aumentada, plataformas \textit{cloud} y sistemas de control tradicionales con éxito. La aplicación resultó útil en las pruebas realizadas, y durante las presentaciones con clientes industriales el \textit{feedback} fue positivo. El trabajo demostró que las integraciones de sistemas de control con las tecnologías 4.0 son posibles, pero es necesario encontrar casos de uso prácticos que justifiquen el esfuerzo del desarrollo. La mano de obra no solo puede automatizarse, sino que también puede mejorarse con ayuda de las nuevas tecnologías.

%----------------------------------------------------------------------------------------
%	SECTION 2
%----------------------------------------------------------------------------------------
\section{Próximos pasos}

El siguiente paso es utilizar un \textit{token} para autenticar la comunicación y certificados TLS  para encriptar el canal de comunicación entre el Hololens 2 y el servidor web. De esta manera se mejoraría la seguridad considerablemente. Utilizando además lo desarrollado, se tiene el punto de partida para realizar una segunda aplicación orientada al mantenimiento de activos en la planta.
